\section{Metodolog\'ia}

A continuaci\'on se describen los aspectos importantes de la metodolog\'ia de este trabajo.

\begin{enumerate}
  \item \textbf{Tipo de investigaci\'on.}
	Este proyecto es de caracter bibliogr\'afico ilustrativo.
   \begin{itemize}
      \item[1.1] \textbf{Bibliogr\'afico:}\\
	    Se har\'a una recopilaci\'on de libros impresos, libros obtenidos en internet, tesis de grado y art\'iculos en linea
	    para contar con el material suficiente que cubra las necesidades del estudio y las que puedan surgir m\'as adelante.
	    El objetivo es compilar de forma coherente la informaci\'on mas \'util y destacada de los resultados que se aplicar\'an
	    en el tema en cuesti\'on.
      \item[1.2] \textbf{Ilustrativa:}\\
	    Ya que se pretende estudiar diversas aplicaciones con la la teor\'ia que se presentar\'a en este texto y en base a ello hacer uso
	    de las herramientas que nos proporcione \'esta para finalmente {\it hacer uso de sofware} (MATLAB \'O libre) y aplicar \'esto a 
	    los problemas que se pretendan estudiar y de esta manera tener una mejor visi\'on de lo importante que es la optimizaci\'on convexa
	    y los beneficios que nos genera.
   \end{itemize}
   \item \textbf{Forma de trabajo:}\\
	 Se tendr\'an reuniones peri\'odicas (una \'o dos veces por semana) con el asesor para tratar los aspectos de la 
	 investigaci\'on tales como analizar y estudiar la teor\'ia, tratar los diferentes aspectos del trabajo escrito y las 
	 presentaciones de avances.
   \item \textbf{Exposiciones.}\\
	 Se tendr\'an dos exposiciones:
	 \begin{itemize}
	    \item \textbf{Primera exposici\'on (p\'ublica):} Presentaci\'on del Perfil del Proyecto de Investigaci\'on.
	    \item \textbf{Segunda exposici\'on (p\'ublica):} Presentaci\'on final del Trabajo de Investigaci\'on: resumen de resultados
		  y aplicaciones.
	 \end{itemize}
\end{enumerate}




