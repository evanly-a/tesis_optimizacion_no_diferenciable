\subsection{Dualidad lagrangiana}


Dado un problema de programacion no lineal hay otro problema de programaci\'on no lineal asociado con el, el primero es llamado {\it problema 
primal,} y el otro es llamdo {\it problema dual lagrangiano.} Bajo ciertas condicines de convexidad y restricciones adecuadas, los 
problemas dual y primal poseen valores objetivos \'optimos iguales, por lo tanto, es posible resolver el problema primal de manera indirecta
resolviendo el problema dual \cite{no-lineal}.\\ \medskip

Considere el siguiente problema de programaci\'on no lineal, el cual es llamado \textbf{\it problema primal}
\medskip

\textbf{\itshape Problema primal P:}
\begin{eqnarray*}
   \mbox{minimizar  } & f(x) & \,\, \\
   \mbox{sujeto a  } & g_i(x) \leqslant 0 & \, \mbox{ para }\,\, i=1, \ldots , m\\
   &  h_i(x) =  0 &\, \mbox{ para }\,\, i=1, \ldots , l\\
   & x \in  X & \,\,
\end{eqnarray*}
\medskip

El {\it problema dual lagrangiano} se indica acontinuaci\'on.\medskip

\textbf{\itshape Problema dual lagrangiano}
\begin{eqnarray*}
   \mbox{maximizar  } & \theta(u, v) & \,\, \\
   \mbox{sujeto a  } & u \geqslant 0 & \,\, \\
   \mbox{donde  } &  \theta(u, v)  = & \inf {f(x) + \sum_{i=1}^{m}u_i g_i(x) + \sum_{i = 1}{l} v_i h_i(x):\, x \in X}
\end{eqnarray*}

Note que la funci\'on dual lagrangiana $\theta$ puede tomar el valor de $- \infty$ para algunos vectores $(u, v).$ Como el problema dual 
consiste en maximizar el \'infimo (la mayor de las cotas inferiores) de la funci\'on  $ f(x) + \displaystyle{\sum_{i=1}^{m}u_ig_i(x) + 
\sum_{i=1}^{l}v_ih_i(x)},$ por lo cual a veces se refieren a \'este como {\it dual viable.}
%agregar interpretacion geometrica del problema dual
\medskip

El siguiente teorema muestra que el valor objetivo de cualquier soluci\'on factible para el problema dual produce una cota inferior en el 
valor objetivo de cualquier soluci\'on factible para el problema primal.

{\teorema \textbf{\itshape Dualidad d\'ebil}\\
Sea $x$ una soluci\'on factible para el problema $P$; esto es $x \in X,\,\, g(x) \leqslant 0\,$ y $\, h(x) = 0. $ Sea $(u, v)$ una soluci\'on
factible para el problema $D; $ esto es $ u \geqslant 0. $ Entonces $f(x) \geqslant \theta(u, v).$}









