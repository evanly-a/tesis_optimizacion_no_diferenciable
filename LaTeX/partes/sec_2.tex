\section{Secci\'on II: Condiciones de optimalidad y dualidad}

% \subsection{Problemas con restricciones de desigualdad}%de desigualdad
% 
% % Un problema sin restricciones es un problema de la forma: $\min f(x)$ sin cualquier restricci\'on al vector $\vec{x}.$ Los problemas sin
% % restricciones rara vez surgen en aplicaciones pr\'aticas. Por lo tanto, se consideran tales problemas aqui porque las condiciones de 
% % optimalidad para problemas con restricciones son una extensi\'ion l\'ogica de las codiciones para problemas sin restricciones.\\ \\
% % 
% % {\definicion Considere el problema de minimizar
% % 
% % \[\min_{x \in \mathbb{R}^n} f(x)\]
% % 
% % y sea $\overline{x} \in \mathbb{R}^n.$ Si $f(\overline{x}) \leqslant f(x)\,\,\, \forall \,x \in \mathbb{R}^n,\,\, \overline{x} $ es llamado 
% % m\'inimo local} \medskip
% 
% En esta subsecci\'on se desarrollar\'a una condici\'on necesaria de optimalidad para minimizar $f(x)$ sujeteo a $x \in S.$ M\'as tarde $S$ 
% ser\'a espec\'ificamente definido como regi\'on factible de un problema de programaci\'on no lineal de la forma
% $\displaystyle{\min_{g(x)\leqslant 0} f(x)}$ y $x \in X.$\\ \\
% \
% 
% \textbf{Condiciones geom\'etrica de optimalidad}\medskip
% 
% {\definicion Sea $S$ un conjunto no vac\'io de $\mathbb{R}^n,$ y sea $\overline{x} \in \overline{S}.$ El cono de direcci\'on factible de $S$ 
% en $\overline{x}$ denotadao por D, est\'a dado por 
% 
% \[D = \{ d: d \neq 0,\,\mbox{ y }\,\,\, \overline{x} + \lambda d \in S\,\,\, \forall \, \lambda \in (0, \delta)\, \,\mbox{para alg\'un }\,
% \delta >0\}\]
% 
% Cada vector $\vec{d} \in D$ no nulo es llamado \textbf{\itshape direcci\'on factible.} En otras palabras, dada una funci\'on
% $f: \mathbb{R}^n \longmapsto \mathbb{R}$, el \textbf{\itshape cono de mejor direcci\'on} en $\overline{x}$ denotado por $F$ est\'a dado por:
% 
% \[F = \{ d: f(\overline{x} + \lambda d) < f(x)\,\, \mbox{ para todo }\, \lambda \in (0, \delta)\,\, \mbox{ para alg\'un }\, \delta > 0 \}\]
% 
% Cada direcci\'on $d \in F$ es llamada \textbf{\itshape direcci\'on  mejorada} \'o \textbf{\itshape direcci\'on descendente} de $f$ en
% $\overline{x}.$ \label{direcciones}}\medskip
% 
% De \'esta definici\'on es claro que un peque\~no movimiento de $\overline{x}$ a lo largo de un vector $d \in D$ conduce a puntos factibles,
% mientras que un movimiento similar a lo largo del vector $d \in F$ conduce a soluciones de mejor valor objetivo.
% \medskip
% 
% {\teorema Considere el problema $\displaystyle{\min_{x \in S} f(x)}$ donde $f:\mathbb{R}^n \longmapsto \mathbb{R}$ y $S$ es un conjunto no 
% vac\'io en $\mathbb{R}^n.$
% \begin{itemize}
%    \item Suponga que $f$ es diferenciable en el punto $\overline{x} \in S.$ Si $\overline{x}$ es una soluci\'on \'optima local, 
% 	 $F_0 \cap D = \emptyset,$ donde $F_0 = \{d: \nabla f(\overline{x})^t d < 0\}$ y $D$ es el cono de direcci\'on mejorada de $S$ en
% 	 $\overline{x}.$
%    \item Inversamente, Suponga que $F_0 \cap D = \emptyset,$
% \end{itemize}
% }

\subsection{Condicion de optimalidad de Fritz John}

{\teorema \textbf{\itshape Condiciones necesarias de Fritz John \cite{no-lineal}}\\
Sea X un conjunto abierto no vac\'io en $\mathbb{R}^n$ y sea $f: \mathbb{R}^n \longmapsto \mathbb{R}$ y 
$g_i : \mathbb{R}^n \longmapsto \mathbb{R}$ para $i = 1, \ldots , m.$ Considere el  Problema P para minimizar $f(x)$ sujeto a $x \in X$ y 
$g_i(x) \leqslant 0$ para $i = 1, \ldots , m.$ Sea $\overline{x}$ una soluci\'on factible  y denotada por $I = \{i: g_(\overline{x}) = 0\}.$
Por lo tanto, suponga que $f $ y $ g_i $ para $ i \in I $son diferenciables en $ \overline{x} $ y que $ g_i $ para $ i \notin I $ son
continuas en $\overline{x}. $ Si $ \overline{x} $ resuelve el problema P localmente, existen escalares $u_0 $ y $u_i$ para $\i \in I$ tales 
que:

\begin{eqnarray*}
   u_0 \nabla f(\overline{x}) +  \sum_{i \in I} u_i \nabla g_i (\overline{x}) & = & 0\\
   u_0, u_i & \geqslant  0  & \mbox{ para  } i\in I\\
   (u_0, u_I) & \neq & (0, 0)
\end{eqnarray*}

donde $ u_I $ es el vector cuyas componentes sin $u_i$ para $i \in I.$ Por lo tanto, si $g_i$ para $i \notin I$ casi son diferenciables en
$\overline{x}$, las codiciones anteriores pueden escribirse de forma equivalente:

\begin{eqnarray*}
   u_0 \nabla f(\overline{x}) +  \sum_{i =1}^{m} u_i \nabla g_i (\overline{x}) & = & 0\\
   u_i g_i(\overline{x}) & = & 0\,\,\,\, \mbox{ para  }\, i=1, \ldots , m\\
   u_o, u_i & \geqslant & 0\,\,\, \mbox{ para  }\, i=1, \ldots , m\\
   (u_0, u) & \neq & (0, 0)
\end{eqnarray*}

donde $u$ es el vector cuyas componentes son $u_i $ para $ i=1, \ldots , m. $ \label{fritz-nec}}
\medskip

{\teorema \textbf{\itshape Condiciones suficientes de Fritz John \cite{no-lineal}}\\
Sea $X$ un conjunto abierto no vac\'io en $\mathbb{R}^n$ y sea $f: \mathbb{R}^n \longmapsto \mathbb{R}$ y 
$g_i : \mathbb{R}^n \longmapsto \mathbb{R}$ para $i = 1, \ldots , m.$ Considere el  Problema P para minimizar $f(x)$ sujeto a $x \in X$ y 
$g_i(x) \leqslant 0$ para $i = 1, \ldots , m.$ Sea $\overline{x}$ una soluci\'on FJ denotada por $I = \{i: g_i(\overline{x}) = 0\}.$ Se
define $S$ como la regi\'on relajada y factible par el problema $P$ en el que se eliminan las restricciones no vinculantes.% FJ=sol de 
%condiciones necesarias

\begin{itemize}%definir seudoconvexas en seccion 4.4
   \item[a.] Si existe un $\varepsilon-$vecindario $N_{\varepsilon}(\overline{x}),\,\,\, \varepsilon > 0,$ tal que $f$ es seudoconvexa sobre 
   $N_{\varepsilon}(\overline{x}) \cap S,\,\, \overline{x} $ es un m\'inimo local para el problema $P.$
   \item[b.] Si $f$ es seudoconvexa en $\overline{x} $ y si $g_i,\,\, i \in I$ ambas son estrictamente pseudoconvexas y cuasiconvexas en 
   $\overline{x}, $ entonces $\overline{x} $ es una soluc\'ion \'optima global para el problema $P.$ En particular, si \'estas suposiciones de
   convexidad generalizada s\'olo son v\'alidas restringiendo el dominio de $f$ para $N_{\varepsilon}(\overline{x}) $ para alg\'un 
   $\varepsilon > 0,\,\,\, \overline{x}$ es un m\'inimo local para el problema $P.$
   \end{itemize}
\label{fritz-suf}}

\medskip

\textbf{Condiciones de Karush-Kuhn-Thucker \cite{no-lineal}}\\ \\ 

{\teorema \textbf{\itshape (Condiciones necesarias de Karush-Kuhn-Thucker KKT)}\\
Sea $X$ un conjunto abierto no vac\'io en $\mathbb{R}^n$ y sea $f: \mathbb{R}^n \longmapsto \mathbb{R} \mbox{ y }
g_i: \mathbb{R}^n \longmapsto \mathbb{R}\,\, \mbox{ para }\, i = 1, \ldots , m.$ Considere el problema $P$ para minimizar $f(x)$  sujeto a
$x\in X$ y $g_i(x) \leqslant 0,\,  \mbox{ para }\, i = 1, \ldots , m.$ Sea $\overline{x}$ una soluci\'on factible y denotada por 
$I = \{i: g_i(\overline{x}) = 0\}.$ Suponga que $f$ y $g_i$ para $i \in I$ son diferenciables en $\overline{x}$ y que $g_i$ para $i \notin I$
son continuas en $\overline{x}.$ Por lanto, suponga que $\nabla g_i(\overline{x})$ para $i \in I$ son linealmente independientes. Si 
$\overline{x}$ resuelve localmente el problema $P,$ existen escalares $u_i$ para $i \in I$ tal que 

\begin{eqnarray*}
   \nabla f(\overline{x}) + \displaystyle{\sum_{i \in I} \nabla g_i(\overline{x})} & = & 0\\
   u_i & \geqslant & 0\,\, \mbox{ para } \, i \in I
\end{eqnarray*}

Adem\'as de las suposiciones anteriores, si para cada $g_i$ con $i \notin I$ es casi diferenciable en $\overline{x},$ las condiciones 
anteriores pueden escribirse de forma equivalente como:

\begin{eqnarray*}
   \nabla f(\overline{x}) + \displaystyle{\sum_{i = 1}^{m} \nabla g_i(\overline{x})} & = & 0\\
   u_ig_i(\overline{x})  & = & 0\,\, \mbox{ para } \, i \in I\\
   u_i & \geqslant & \,\, \mbox{ para } \, i \in I
\end{eqnarray*} \label{KKT}}
\medskip

\subsection{Problemas con restricciones de igualdad y desigualdades}

Acontinuaci\'on se generalizar\'a  las condiciones de optimalidad vistos anteriormente, para ello se sonsidera el siguiente problema de 
programaci\'on no lineal $P:$

\begin{eqnarray*}
   \min f(x) &  \,\, &  \\
   \mbox{sujeto a }\, g_i(x) & \leqslant & 0\,\,\, \mbox{ para }\, \, i=1, \ldots ,m\\
   h_i(x) & = & 0\,\, \mbox{ para }\, \, i=1, \ldots ,l\\
   x & \in & X 
\end{eqnarray*}


			%Problemas con restricciones de igualdad y desigualdades
\subsection{Condici\'on suficiente y necesaria para problemas con restricciones de segundo orden}
~\\

Basados en varias suposiciones de convexidad (generalizada), se tienen condiciones suficientes para garantizar que dada una soluci\'on que
satisface las condiciones de \'optimalidad de primer orden \'esta es local o glabamente \'optima. An\'alogo al caso sin restricciones, ahora se
sabe que la derivada de segundo orden es una condici\'on necesaria y suficiente para el problema restringido.\cite{no-lineal}\\
Considere el problema:

\begin{subequations}
   \begin{equation}
      P: \min \{f(x): x \in S\}
   \end{equation}
   \mbox{donde}
   \begin{equation}
      S = \{x: g_i(x) \leqslant 0\,\, \mbox{ para }\,\, i=1, \ldots , m,\,\,\, h_i(x) = 0\,\, \mbox{ para }\,\,  i=1, \ldots , l;\,\, x \in X\}
   \end{equation}
\end{subequations}

~\medskip

Asuma que $f, g_i$ para $i=1, \ldots , m $ y $ h_i $ para $ i=1, \ldots , l $ est\'an definidas en $\mathbb{R}^n \longmapsto \mathbb{R}, $ 
ambas son diferenciables y $X$ es un conjunto abierto no vac\'io en $\mathbb{R}^n.$ La {\it funci\'on lagrangiana} para este problema se
define como:

\begin{equation}
   \phi (x, u, v) = f(x) + \displaystyle{\sum_{i=1}^{m} u_i g_i(x) + \sum_{i=1}^{l} v_i h_i(x)}
\end{equation}
\medskip

Esta funci\'on permite formular una teor\'ia de dualidad para problemas de programaci\'on no lineal.




			%Condici\'on suficiente y necesaria para problemas con restricciones de segundo orden
\subsection{Dualidad lagrangiana}


Dado un problema de programacion no lineal hay otro problema de programaci\'on no lineal asociado con el, el primero es llamado {\it problema 
primal,} y el otro es llamdo {\it problema dual lagrangiano.} Bajo ciertas condicines de convexidad y restricciones adecuadas, los 
problemas dual y primal poseen valores objetivos \'optimos iguales, por lo tanto, es posible resolver el problema primal de manera indirecta
resolviendo el problema dual \cite{no-lineal}.\\ \medskip

Considere el siguiente problema de programaci\'on no lineal, el cual es llamado \textbf{\it problema primal}
\medskip

\textbf{\itshape Problema primal P:}
\begin{eqnarray*}
   \mbox{minimizar  } & f(x) & \,\, \\
   \mbox{sujeto a  } & g_i(x) \leqslant 0 & \, \mbox{ para }\,\, i=1, \ldots , m\\
   &  h_i(x) =  0 &\, \mbox{ para }\,\, i=1, \ldots , l\\
   & x \in  X & \,\,
\end{eqnarray*}
\medskip

El {\it problema dual lagrangiano} se indica acontinuaci\'on.\medskip

\textbf{\itshape Problema dual lagrangiano}
\begin{eqnarray*}
   \mbox{maximizar  } & \theta(u, v) & \,\, \\
   \mbox{sujeto a  } & u \geqslant 0 & \,\, \\
   \mbox{donde  } &  \theta(u, v)  = & \inf {f(x) + \sum_{i=1}^{m}u_i g_i(x) + \sum_{i = 1}{l} v_i h_i(x):\, x \in X}
\end{eqnarray*}

Note que la funci\'on dual lagrangiana $\theta$ puede tomar el valor de $- \infty$ para algunos vectores $(u, v).$ Como el problema dual 
consiste en maximizar el \'infimo (la mayor de las cotas inferiores) de la funci\'on  $ f(x) + \displaystyle{\sum_{i=1}^{m}u_ig_i(x) + 
\sum_{i=1}^{l}v_ih_i(x)},$ por lo cual a veces se refieren a \'este como {\it dual viable.}
%agregar interpretacion geometrica del problema dual
\medskip

El siguiente teorema muestra que el valor objetivo de cualquier soluci\'on factible para el problema dual produce una cota inferior en el 
valor objetivo de cualquier soluci\'on factible para el problema primal.

{\teorema \textbf{\itshape Dualidad d\'ebil}\\
Sea $x$ una soluci\'on factible para el problema $P$; esto es $x \in X,\,\, g(x) \leqslant 0\,$ y $\, h(x) = 0. $ Sea $(u, v)$ una soluci\'on
factible para el problema $D; $ esto es $ u \geqslant 0. $ Entonces $f(x) \geqslant \theta(u, v).$}









			%Dualidad lagrangiana



