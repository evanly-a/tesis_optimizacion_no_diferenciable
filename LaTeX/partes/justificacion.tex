\section{Justificaci\'on}


En nuestra realidad, vivimos con ciertas comodidades y a medida que han transcurrido los a\~nos  los conocimientos han avanzado tanto
que hoy en d\'ia poseemos un sistema n\'umerico universal, celulares, computadoras, internet, tarjetas de cr\'edito, 
tarjetas de d\'ebito, identificaciones, cohetes espaciales, etc. Pero todos \'estos pusieron un reto a la humanidad y para poder
resolverlo no solo observamos que ten\'iamos un problema sino que lo planteamos y buscamos una soluci\'on a \'este. Claramente
esto requiere de tiempo y conocimiento, debido a ello es importante que sepamos y resaltemos que todo tiene una base matem\'atica.\\ \\

En medicina por ejemplo, cuando se crean curas para enfermedades se deben tomar desiciones sobre que elementos qu\'imicos utilizar porque 
unos son m\'as agresivos que otros.\\

Podemos decir que la optimizaci\'on matem\'atica es el estudio sobre c\'omo hacer la mejor elecci\'on cuando estamos limitados por un
conjunto de requerimientos. \\  

En cultivos (cualquier tipo de cultivo) se debe saber que tipo de fertilizantes e insecticidas usar y la cantidad correcta, de lo contrario
eso resultar\'ia en una mala cosecha o la muerte del cultivo.\\

Eso es solo por mencionar algunos ejemplos, lo que hacemos est\'a ligado a las matem\'aticas y en cuanto a toma de desiciones no solo podemos
estudiar estad\'istica sino que tambien la optimizaci\'on matem\'atica se encuentra inmersa.\\ \\

Ahora bien, sabemos que en las matem\'aticas trabajamos con espacios y funciones te\'oricas, por as\'i decirlo, pero: que pasa cuando aplicamos 
nuestros conocimientos a la realidad? si bien es cierto que nuestro espacio y todas sus maravillosas propiedades se mantienen. Cuando hablamos
en concreto de optimizaci\'on convexa (cuya base te\'orica es el an\'alisis convexo) nos referimos a minimizar funciones convexas reales 
definidas para una variable contenida dentro de un subconjunto convexo de un espacio vectorial. \\

Las funciones no diferenciables en todas partes, marcaron un paradigma en la teor\'ia moderna de optimizaci\'on \cite{intro}. Se encontr\'o 
que muchos problemas de optimizaci\'on convexa no eran diferenciables en el punto m\'inimo y aqu\'i es donde surge la necesidad e importancia
de estudiar Optimizaci\'on Convexa no Diferenciable ya que debido a ella no solo se desaroll\'o nueva teor\'ia matem\'atica sino que
tambi\'en las t\'ecnicas abarcadas por este campo de estudio son importantes en aplicaciones de ingenier\'ia (problemas de gesti\'on 
hidrot\'ermica, problemas de control \'optimo de producci\'on, reconstrucci\'on \'optima de im\'agenes tomogr\'aficas, problemas de ruteo
en redes de transporte y telecomunicaciones.) porque al plantear un problema de diseño en forma convexa se puede identificar la estructura de 
la solución \'optima, que suele revelar aspectos importantes de dicho diseño.


% As\'i, un enfoque completamente diferente se desarroll\'o,
% en donde fue concebida la noci\'on de subdiferencial. El subdiferencial de una funci\'on en un punto dado es un conjunto que se puede 
% considerar como un sustituto de la noci\'on de la derivada en puntos donde la funci\'on no es diferenciable. La importancia del 
% subdiferencial fue que un conjunto de reglas de c\'alculo pudo ser desarrollado, y que se volvi\'o muy \'util para llevar a cabo el 
% an\'alisis de funciones no diferenciables. Fue entonces natural extender el dominio de optimizaci\'on no diferenciable mas all\'a de la 
% convexidad. \\ \\

% 
% Esto se debe, entre otras cosas al hecho de que cualquier soluci\'on local es tambi\'en una soluci\'on global, ya que existe una teor\'ia 
% de dualidad y unas condiciones de punto \'optimo que permiten verificar la soluci\'on encontrada.\\ \\

% En lo que se refiere a la teor\'ia, ahora es habitual considerar hechos b\'asicos del an\'alisis convexo para toda la clase de funciones
% convexas, incluyendo funciones no diferenciables.
